% Term Paper
% Marcus Ortiz
% CPE-300
% October 20, 2011

\documentclass[11pt]{article}

\usepackage{setspace}
\usepackage{url}
\usepackage{geometry}
\usepackage{multicol}

\geometry{letterpaper}
\begin{document}

% Title
\title{\vfill Firesheep: Burning Through Web Security}
\author{
Marcus Ortiz\vspace{10pt} \\
CPE 300\vspace{10pt} \\
\vspace{10pt} \\
}
\date{October 20, 2011}
\maketitle

% Abstract
\vfill
\begin{abstract}
On October 24, 2010, a software developer from Seattle named Eric Butler released an open source add-on for the Firefox web browser named \emph{Firesheep}. This add-on allows for users to capture \emph{and use} unencrypted login credentials for anyone on the same open wifi network within two simple mouse clicks \cite{codebutler_main}. Although Butler's intentions with this project are to provide awareness to the urgent need for encrypted web sessions, this clearly presents a moral dilemma to professionals in the computer science and software engineering fields. Can the decision to release this software to the public possibly be considered an ethical one?

By looking to the  ACM and IEEE-CS approved \emph{Software Engineering Code of Ethics and Professional Practice} \cite{se_code} and various other sources, the case can be made that this choice was in fact an ethical one. The \emph{Firesheep} project acts in the interest of the public good by revealing the serious security flaws that exist in many of the Internet's most visited websites \cite{alexa} \cite{github}. It urges the adoption of safer practices to be taken by people using public wifi and promotes action to be taken by companies to secure their site's users, which will ultimately result a better web for everyone.
\end{abstract}

% Table of Contents
\thispagestyle{empty}
\newpage
\tableofcontents
\thispagestyle{empty}
\newpage

% Paper Content
\begin{multicols}{2}
\setcounter{page}{1}

\section{Background}

% Bibliography
\end{multicols}
\newpage
\nocite{*}
\bibliographystyle{IEEEannot}
\bibliography{termpaper}
\end{document}
