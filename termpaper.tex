% Term Paper
% Marcus Ortiz
% CPE-300
% October 20, 2011

\documentclass[11pt]{article}

\usepackage{setspace}
\usepackage{url}
\usepackage{geometry}
\usepackage{multicol}

\geometry{letterpaper}
\begin{document}

% Title
\title{\vfill Firesheep: Helping the Users Win}
\author{
Marcus Ortiz\vspace{10pt} \\
CPE 300\vspace{10pt} \\
\vspace{10pt} \\
}
\date{October 20, 2011}
\maketitle

% Abstract
\vfill
\begin{abstract}
On October 24, 2010, a software developer from Seattle named Eric Butler released an open source add-on for the Firefox web browser named \emph{Firesheep}. This add-on allows for users to capture \emph{and use} unencrypted login credentials for anyone on the same open wifi network within two simple mouse clicks \cite{codebutler_main}. Although Butler's intentions with this project are to provide awareness to the urgent need for encrypted web sessions, this clearly presents a moral dilemma to professionals in the computer science and software engineering fields. Can the decision to release this software to the public possibly be considered an ethical one?

By looking to the ACM and IEEE-CS approved \emph{Software Engineering Code of Ethics and Professional Practice} \cite{se_code} and various other sources, the case can be made that this choice was in fact an ethical one. The \emph{Firesheep} project acts in the interest of the public good by revealing the serious security flaws that exist in many of the Internet's most visited websites \cite{alexa} \cite{github}. It urges the adoption of safer practices to be taken by people using public wifi and promotes action to be taken by companies to secure their site's users, which will ultimately result a better web for everyone.
\end{abstract}

% Table of Contents
\thispagestyle{empty}
\newpage
\tableofcontents
\thispagestyle{empty}
\newpage

% Paper Content
\begin{multicols}{2}
\setcounter{page}{1}

\section{Known Facts}
\subsection{Background}
Last October, a freelance web application and software developer by the name of Eric Butler released a tool called \emph{Firesheep} to the world during the 12th annual \emph{Toorcon} conference in San Diego \cite{codebutler_main}. \emph{Firesheep} provides a graphical user interface that allows users to "hijack" web sessions from others on the same public wifi network. Toorcon is an event held in different cities that brings together some of the top experts on security to learn and discuss the state of the field among peers \cite{toorcon}. Releasing this software at \emph{Toorcon} was not an accident--Butler presented a talk to go along with the release of \emph{Firesheep} in which he describes his purpose for creating it. His intent was to call attention to the security weaknesses of many websites in an effort to promote improved security for these websites and the web in general. This controversial tool fittingly set the world of web security ablaze within weeks of its release. Only a year later, and \emph{Firesheep} already has 1,925,376 downloads \cite{github}. Public reaction to the software was highly polarized and continues to be.

\subsection{Firesheep: What?}
Butler's tool was developed as a free, open source add-on for the popular web browser \emph{Firefox}. Once the add-on has been installed on a user's machine, it can be accessed directly through the web browser as a sidebar. Opening this sidebar will provide the user with the option to "Start Capturing". When a user presses this button, the tool will begin scanning the web traffic from all other computers that happen to be using the same public wifi network. Whenever one of these computers accesses a website that \emph{Firesheep} supports, the name and photo of the user interacting with the site will appear in the sidebar. Here is where it gets interesting: if one of these photos is double-clicked, then the \emph{Firesheep} user will be logged into that site with the credentials of the user from the other computer. \cite{codebutler_main}

As an example, let's assume that user Alice has installed \emph{Firesheep} on her computer and is using it. User Bob happens to be using the same wifi network as Alice, and he logs onto the http destination for Facebook. Alice will now see Bob's name and profile photo on her computer. When she double-clicks on this, she'll be taken to Facebook and logged in as Bob on her computer. Alice has now effectively stolen Bob's facebook identity with a few simple clicks. This process is known as "session hijacking" \cite{hijacking}.

\subsection{Firesheep: How?}
Many websites require users to sign in to use their personalized version of the site. To accomplish this, the website sends your computer a "cookie" when you sign in with the right username and password combination. These (unfortunately inedible) cookies are text files that are used to identify the user's session on the site \cite{codebutler_main}. Unless this website is using encryption technologies, these cookies can be easily intercepted and read across any public wifi network with the help of a "packet sniffer". This is the core technology that \emph{Firesheep} is based on.

The backend for the tool is written in C++ and uses 3rd party packet capturing software \cite{github}. Packet capturing software is not new and has been around for a while--\emph{Firesheep} parses these packets in a way that reveals the information about cookies that can be found from them. Another component of the tool, called "handlers", allows other developers to add support for specific websites by hooking into a minimal javascript api \cite{github}. In this way, the project is very community based and has potential to target any website that isn't properly protected. As of the the time that this paper is being written, there are 31 well-known sites that have "handlers" associated with them \cite{github}. Some of these sites are the most popular in existence \cite{alexa}.

\subsection{Firesheep: Why?}
Butler's aim with this project is to create a greater public awareness of the general lack of security that has permeated some of the world's most visited websites. Specifically, he believes that \emph{the only effective fix for this problem is full end-to-end encryption, known on the web as HTTPS or SSL} \cite{codebutler_main}. His purpose in releasing \emph{Firesheep} to the public is to show how easily "session hijacking" can be achieved, so that the need for SSL encryption will be realized and action will be taken to better protect the privacy of internet users. In his own words, \emph{it's time for everyone to demand a more secure web} \cite{codebutler_main}.

Surprisingly, my research into discovering specific instances in which the use of \emph{Firesheep} has caused damages has yet to produce any results. There haven't been any court cases to come up in the past year that have any reference to the software, yet it seems unlikely that this will continue to be true. I even found it difficult to find any blog posts from people who have been directly affected by another person's use of the tool. This hardly means that \emph{Firesheep} has yet to be used maliciously, though there seems to be very little documented incidents currently. 

\section{Research Question}
Was Eric Butler's decision to release \emph{Firesheep} an ethical one?

This question is important, because it deals with the flawed security of sites that the average person uses everyday. The risks of poor security heavily affect these people. Facebook, for example, is the 2nd most visited site in the world \cite{alexa}. It is also a supported site with \emph{Firesheep} \cite{github}. Some people would argue that it isn't acceptable to release a tool that makes it so easy to access such private content. On the other hand,  some people would point out that is a necessary step to get these companies like Facebook to focus on protecting their users' privacy.

\section{Arguments For}
\subsection{Helping the Users Win}
As Eric Butler  puts it, \emph{my hope is that Firesheep will help the users win} \cite{codebutler_main}. His major aim in releasing the software is to expose the vulnerability of popular websites, which, he hopes, will encourage these sites to improve the experiences for their users. In his opinion, \emph{websites have a responsibility to protect the people who depend on their services} \cite{codebutler_main}.

Currently, there are several factors that contribute to a website's insecurity: \cite{codebutler_blog_1}
\begin{itemize}
  \item
  the lack of https/ssl encryption for the site
  \item
  forcing users to pay for https/ssl encryption
  \item
  using https/ssl encryption only for posting login credentials
  \item
  flawed use of https/ssl encryption
\end{itemize}

When a major website is lacking in one of these areas, the users of that website can potentially be exploited. Thus, by implementing a handler for this site using \emph{Firesheep}, this problem can get the attention that it deserves and therefore must be addressed by the site's owners. Eric Butler believes that the use of his tool will show the users of these websites just how vulnerable they really are, which will in turn lead to a growing public outcry at this fact. He argues that this public outcry will eventually lead to users demanding major improvements from these sites, thereby justifying the release of the tool \cite{toorcon_slides}.

There are several examples of this happening in the weeks after \emph{Firesheep} was released. GitHub, a website for social coding, initially was in the category of only using https/ssl encryption for paying users. Upon \emph{Firesheep}'s release, GitHub became immediately aware of the potential harm that could come to their users by not improving their security measures. In a series of three blog posts, the company describes how they acted quickly to provide full end-to-end encryption for both paying and non-paying users as a direct response to the awareness created by Butler's tool \cite{github_reaction}.

Another major website to respond to \emph{Firesheep} was Facebook. Although they weren't as quick to react as GitHub, they too realized the significance of better protecting their users due to the publicity created by Butler's tool. In January 2011, Facebook implemented site-wide https/ssl encryption for their users \cite{facebook_reaction}.

Twitter was another huge website that previously only used https/ssl encryption for logging in users. As of March 2011, Twitter has followed in the footsteps of GitHub and Facebook in providing their users with the ability to fully encrypt their sessions with the site \cite{twitter_reaction}.

These are only some of the most publicized instances of \emph{Firesheep}'s positive influence on web security. Given these specific cases, it can be argued that Butler's goals with the project have been largely achieved. This greatly improved public awareness supports the claim that Butler's release of the tool was indeed ethical.

\section{Arguments Against}
\subsection{Breaking the Law}
There is also an argument to be made that Eric Butler's decision to release \emph{Firesheep} was \emph{un}ethical. A major support for this argument comes from the many troubling legal issues that the software could potentially be involved in. The first major law that the use of \emph{Firesheep} could potentially conflict with is titled \emph{Fraud and Related Activity in Connection With Computers} (Title 18, Part 1, Chapter 47, Section 1030) \cite{law_1}. Joseph V. DeMarco, a former assistant U.S. attorney who \emph{specializes in representing and counseling clients on complex issues involving information privacy and security} \cite{legal_forbes} \cite{demarco}, claims this:

\begin{quote}
  \emph{...individuals who use the extension to access the accounts of others without those persons’ knowledge or consent are almost certainly violating the computer trespass provisions of the Computer Fraud and Abuse Act and are also potentially engaging in an unlawful data tap...} \cite{legal_forbes}
\end{quote}

Although DeMarco does not come right out and say that releasing \emph{Firesheep} was unethical (and no other statement has been found to say this either), this general argument can be inferred from the fact that \emph{Firesheep} can easily be used to break this law.

Another law that users of \emph{Firesheep} may be breaking is titled \emph{Interception and Disclosure of Wire, Oral, or Electronic Communications Prohibited} (Title 18, Part 1, Chapter 119, Section 2511) \cite{law_2}. According to Scott Christie, a partner in the Newark, N.J., office of law firm of McCarter \& English \cite{legal_cw}:

\begin{quote}
  \emph{[People] don't forfeit their expectation of privacy simply by using a public Wi-Fi spot. And wiretap laws in general make it illegal to intercept real-time communications and content.} \cite{legal_cw}
\end{quote}

Christie, like DeMarco, believes that the usage of \emph{Firesheep} constitutes illegal activity. Even though they don't explicitly label Butler's release of the tool as unethical, the argument that both men make regarding the law seems to point to this idea.

\subsection{Exploiting Privacy}
Recently, a pair of researchers at Bell Labs conducted an experiment with a version of \emph{Firesheep} that they modified. They wanted to see how it could be used in a potentially harmful way. This experiment involved tracking volunteers' usage of Google search with the tool. They set up \emph{Firesheep} to scan the volunteers on the open network. Then, they used a custom made handler for google.com that in addition to capturing their cookies, also tracked the destinations the users were clicking to from google searches. The two researchers were able to successfully recover up to 80 percent of the participants' personal click history \cite{show_me_your_cookie}.

This proof of concept showed how \emph{Firesheep} can be used to exploit one's privacy. Google.com, the world's most visited site \cite{alexa}, is a website that is used by millions of people on a regular basis. Furthermore, Wifi hotspots are predicted to more than triple by the year 2015 \cite{informa}. This combination makes for an incredibly vulnerable target that \emph{Firesheep} is designed to take advantage of. If this software gets into the wrong hands, it could easily be used to harm a large number of individuals by exposing their search data. With this possible, it seems like labeling the release of \emph{Firesheep} as ethical might be a mistake. Although the Bell Labs researchers did not specifically make this claim, this is an argument that can be inferred from their paper.

\section{Analysis}
\subsection{The SE Code}
This paper will focus on using the ACM and IEEE-CS approved \emph{Software Engineering Code of Ethics and Professional Practice} \cite{se_code} to determine the ethical nature of \emph{Firesheep}'s release to the public. This code of ethics is recognized by IEEE and ACM as being the standard to which professionals in the field should be judged with. The preamble of this document states that \emph{anyone claiming to be or aspiring to be a software engineer} is obligated to follow the guidelines set out by its eight principles and respective codes \cite{se_code}. "Software engineers" are then defined as those \emph{who contribute by \textbf{direct participation} or by teaching, to the analysis, specification, design, \textbf{development}, certification, maintenance and testing of \textbf{software systems}} \cite{se_code}.

I believe that this code of ethics is directly applicable to the case of \emph{Firesheep}. \emph{Firesheep} is an "addon" to the popular web browsing software system Firefox \cite{codebutler_main}, and Eric Butler is the person who developed it. This means that Butler matches the SE code's definition of a software developer, and he should therefore be held accountable to its principles. The specific principles "Public", "Judgment", and "Professions" have been chosen as a means of best analyzing Butler's decision to release \emph{Firesheep}.

\subsection{Public Principle}
The "Public Principle" of the SE code is the first code section in the document and deals with the relationship between software developers and the public to which they interface with in some way that uses their skills as a professional. It is essential in determining whether Butler's decision was ethical or not, as it relates directly to his target audience for \emph{Firesheep}.

\subsubsection{SE Code 1.04}
Eric Butler's purpose with the release of \emph{Firesheep} was to warn users that many of the websites they use heavily are very insecure \cite{codebutler_main} \cite{toorcon_slides}. This goal will be analyzed to determine if it matches up with section 1.04 of the SE code. It states:

\begin{quote}
  \textbf{1.04} [\textbf{software engineers} shall] \emph{disclose to \textbf{appropriate persons} or authorities any \textbf{actual or potential danger to the user}, the public, or the environment, that they reasonably believe to be associated with \textbf{software or related documents}.} \cite{se_code}
\end{quote}

When discussing the purpose behind the creation of \emph{Firesheep}, Butler states \emph{this is a widely known problem that has been talked about to death, yet very popular \textbf{websites} continue to fail at protecting their users} \cite{codebutler_main}. He makes it clear that insecure websites are his main concern regarding the release of the tool. So, when analyzing this code, these \textbf{insecure websites} will be substituted for the \textbf{software or related documents}  that the code mentions.

To make the substitution for the phrase \textbf{appropriate persons} from the code, the term "appropriate" should first be defined. In this case, it makes sense that the "appropriate persons" should be the people who run the websites, because they are the only ones who have the power to change that site. In Butler's words, \emph{Websites have a responsibility to protect the people who depend on their services} \cite{codebutler_main}. By "websites", he is actually referring to \textbf{the people who run them}. In Butler's eyes, they are one appropriate group of people to disclose any information that may relate to the sites. In addition, he feels that \textbf{the people who use these sites} are also an appropriate group of people that he should disclose this information to. This is evident from the talk that he gave at \emph{Toorcon} \cite{toorcon_slides}. The audience there was made up of people who use these websites on a regular basis, and Butler encourages them to be more aware in the slides. He feels that the user of these sites are directly impacted by the behavior of the sites, so it is essential that they have as much knowledge on their behavior as possible. So, when applying SE code 1.04, the phrase \textbf{insecure websites and their users} will be substituted for the term \textbf{appropriate persons}.

The \textbf{potential danger} to users of these websites is discussed in Butler's blog post, which details the problems associated with not using full site-wide https/ssl encryption for sessions. He states:

\begin{quote}
  \emph{It's extremely common for websites to protect your password by encrypting the initial login, but surprisingly uncommon for websites to encrypt everything else. This leaves the cookie (and the user) vulnerable...On an open wireless network, cookies are basically shouted through the air, making these attacks extremely easy.} \cite{codebutler_main}
\end{quote}

According to Eric Butler, the \textbf{potential danger} presented by insecure websites is the possibility that the users of these websites can very easily have their login credentials revealed to others (through the cookies).
By not providing their users with appropriate encryption options, Butler argues that these sites can potentially harm their users. Butler is attacking this problem by using his tool to "disclose" this information. He introduces \emph{Firesheep} on his initial blog post by claiming that it is \emph{designed to demonstrate just how serious this problem is} \cite{codebutler_main}.

For the reader's reference, here is a transformed version of code section 1.04 that directly substitutes the context specific information that is relevant to the \emph{Firesheep} case.

\begin{quote}
  \textbf{1.04} [\textbf{Eric Butler} shall] \emph{disclose to \textbf{insecure websites and their users} any \textbf{potential for having users' login credentials being easily stolen}, that [he] reasonably believe[s] to be associated with \textbf{the insecure websites}.}
\end{quote}

The current state of affairs regarding effective security within the world of major websites is sadly lacking. This is a product of the "process of incompetence" that Peter Farrell-Vinay \cite{incompetence} highlights in his paper. According to Vinay, the poor nature of many software projects are a result of poor management and a company culture where employees are frowned on for pointing out inadequacies that exist in the software. This idea is directly applicable to the world of web development. The vulnerability of unencrypted web sessions is something that these companies are aware of, yet they have been actively ignoring it for the most part. To address this, Eric Butler is using \emph{Firesheep} to fight this "process of incompetence". By helping create a highly publicized awareness of this problem, he is attempting to combat the harm that this has on the users of the web.

By applying this SE code, it becomes evident that releasing \emph{Firesheep} was not only an ethical decision, but also one that needed to be made by Butler as a professional in the field of computing. As stated earlier, his hope is to \emph{help the users win} \cite{codebutler_main}.

\subsection{Judgment Principle}
The "Judgment Principle" from the SE code is also very important in analyzing the ethical nature of this incident. It deals with the judgments that software developers should make regarding their work as professionals. It acts as a guideline to which these professionals can look to in order to best make decisions regarding software that they may be associated with and should provide some insight into whether Butler acted ethically or not in providing \emph{Firesheep}.

\subsubsection{SE Code 4.01}
\begin{quote}
  \emph{Today at Toorcon 12 I announced the release of Firesheep, a Firefox extension designed to demonstrate just how serious this problem is.} \cite{codebutler_main}
\end{quote}

When Eric Butler made the decision to release \emph{Firesheep}, did he consider the impact that this decision would have on others? According to the SE Code of Ethics, this is something that should have been done. It states:


\begin{quote}
\textbf{4.01} [\textbf{software engineers} shall] \emph{temper all \textbf{technical judgments} by the need to support and maintain \textbf{human values}.} \cite{se_code}
\end{quote}

By applying this code, Butler's "technical judgment" in releasing the tool can only be considered ethical if he did so under the realization that his action would "support and maintain human values". There are many "human values" that could be associated with \emph{Firesheep}'s release, but the most applicable of these to this specific situation would be the value that people place on "right to privacy". Privacy of one's own beliefs and possessions is something that most people value. Regarding privacy, the "Electronic Frontier Foundation" states:

\begin{quote}
  \emph{Privacy rights are enshrined in our Constitution for a reason — a thriving democracy requires respect for individuals' autonomy as well as anonymous speech and association.} \cite{eff_privacy}
\end{quote}

The EFF is an international and donor funded non-profit organization founded in 1990 that seeks to defend consumers and the general public. Their stated goal is to \emph{confront cutting-edge issues defending free speech, privacy, innovation, and consumer rights} \cite{eff_about} in the digital age of today. This organization is an established name in the world of computing and should serve as a respected source when it comes to technical matters that have a meaningful impact on human values. From their stated goals, it is clear that they believe that privacy is one of the most important major human values. So--did Eric Butler's decision take this into account?

The existence of \emph{Firesheep} is undoubtedly a potential obstacle to people's privacy. The most prevalent example of this is the website facebook.com. Upwards of 40 percent of global internet users visit facebook daily \cite{alexa}. Users can upload personal photos, send personal messages to other users, and even video chat with other users on the site. As stated earlier, \emph{Firesheep} supports Facebook, which means that a user of the tool could potentially access someone Elise's personal information. This personal information is thought of as being private by the user. At first glance, it appears as if this SE code was overlooked by Butler. The use of \emph{Firesheep} could be thought of as an attack on this basic human value of privacy, since it can expose these personal items.

The purpose of Butler's software, however, is not to attack the privacy of individuals--it is to create an awareness of the lack of privacy provided by these websites. In his blog post introducing \emph{Firesheep}, he claims:

\begin{quote}
\emph{Websites have a responsibility to protect the people who depend on their services. They've been ignoring this responsibility for too long, and it's time for everyone to demand a more secure web. My hope is that Firesheep will help the users win.} \cite{codebutler_main}
\end{quote}

According to the software's creator, his aim is not to threaten privacy. His intent with releasing it is actually to bring about positive change regarding privacy. Butler feels that many top websites simply aren't doing enough to protect the privacy of their users, and this is the only way to get the attention of the people who run these sites. Supporting (and maintaining) the human value of privacy through the act of improving awareness is the ultimate goal of the \emph{Firesheep} project.

The transformed text of code section 4.01 is presented here as a reference for the reader:

\begin{quote}
\textbf{4.01} [\textbf{Eric Butler} shall] \emph{temper \textbf{his decision to release Firesheep} by the need to support and maintain \textbf{people's right to privacy}.} \cite{se_code}
\end{quote}

Unfortunately, this issue is not black and white--there is definitely a moral gray area regarding privacy here. In order to gain a bigger focus on privacy in the future, it seems as if Butler is sacrificing others' more immediate need for privacy. From a strictly utilitarianism perspective, Butler's actions could be seen as an "overall good" that benefits human values due to the long-term outcome of the software's release. On the other hand, his actions could be seen as harmful, especially as it relates to people who are unaware of their lack of privacy on these websites.

Ultimately, Butler \emph{did} have these issues in mind before he released the tool \cite{codebutler_main}. SE Code 4.01 says that software engineers should "temper their judgments" by the need to support these values \cite{se_code}. Eric Butler has indeed done this. His judgment to release \emph{Firesheep} was motivated by his efforts to support the privacy of the general "internet-using" population \cite{codebutler_main}, even if it may come at the cost of disrupting the privacy of a small set of individuals.

\subsection{Profession Principle}
The "Profession Principle" of the SE code exists as a means of conveying how professionals in the software engineering field should behave. It lists out the expected actions that a \emph{professional} software developer should take given certain circumstances. It will be used to analyze whether or not Butler acted as an ethical professional when he released his software at \emph{Toorcon}.

\subsubsection{SE Code 6.12}
Section 6.12 of the code states:
\begin{quote}
  \textbf{6.12} [\textbf{software engineers} shall] \emph{\textbf{express concerns} to the \textbf{people involved} when \textbf{significant violations of this Code} are detected unless this is impossible, counter-productive, or dangerous.} \cite{se_code}
\end{quote}

As Eric Butler stated both in his talk and in his blog post introducing \emph{Firesheep}, there are a number of popular websites that simply aren't doing enough to keep their users' login credentials secure \cite{toorcon_slides} \cite{codebutler_main}. The SE code applies to these websites just as much as it does to Butler. Their inability to properly secure their users is a major violation of this code. Specifically, this violates SE code section \textbf{1.03}, which states that \emph{[software engineers shall] approve software only if they have a well-founded belief that it is safe, meets specifications, passes appropriate tests, and \textbf{does not} diminish quality of life, \textbf{diminish privacy} or harm the environment.} \cite{se_code}. By not providing their users with end-to-end HTTPS encrypted sessions (even when it is a well accepted and encouraged practice), these websites are violating this code. Thus, Butler is following code section \textbf{6.12} by making this fact known to the users of these sites, as they are the "people involved". Butler intends for \emph{Firesheep} to "express concerns" by raising awareness of this issue.

This is the transformed version of the code given the details relevant to this event:

\begin{quote}
  \textbf{6.12} [\textbf{Eric Butler} shall] \emph{\textbf{provide awareness} to the \textbf{users of insecure websites} when \textbf{these websites are diminishing their privacy} unless this is impossible, counter-productive, or dangerous.}
\end{quote}

An important thing to note, however, is that the code ends with \emph{shall express concerns...\textbf{unless this is impossible, counter-productive, or dangerous}}. This is an interesting topic of discussion, because it could be argued that Butler is expressing his concerns in a manner that is "counter-productive". How can he criticize these websites for not paying enough attention on privacy when his tool can be used to exploit this lack of privacy? Again, to settle this debate, I will base my analysis on the ethical system of "Utilitarianism". Utilitarianism holds that the most beneficial decisions made are those that maximize the "overall happiness" \cite{utility}. In short, the outcome of an event is more important than how the event transpired.

For \emph{Firesheep}, this means that if the overall outcome of its release is that the web in general becomes a more secure place (under the safe assumption that more security results in more happiness), then this justifies the few individuals who may have been negatively impacted by the software. With the actions taken by companies like GitHub, Facebook, and Twitter in the past year, this overall beneficial outcome to the web using public seems to be a reality. More and more major websites are gradually realizing the importance of session security and implementing end-to-end encryption for their users. Some of this comes as a direct response to \emph{Firesheep's} release \cite{github_reaction}, while even more has come about as an indirect response to the public's reaction to the tool \cite{facebook_reaction}, \cite{twitter_reaction}. So far, it seems as if \emph{Firesheep} has been able to quickly attack the "process of incompetence" that exists within the web development community regarding the use of HTTPS and SSL encryption technology. This is a major win for the users.

\subsection{Colleagues Principle}
\subsubsection{SE Code 7.06}
As mentioned earlier, the release of \emph{Firesheep} coincided with a presentation given at the \emph{Toorcon} conference in San Diego. \emph{Toorcon} is an conference that is held annually in which people from all over the world attend to discuss and learn about the latest information on security \cite{toorcon}. Here, the top experts on security for computing give presentations on old and new vulnerabilities to watch out for and how to protect systems from them.

Section 7.06 of the SE code states:

\begin{quote}
  \textbf{7.06} [\textbf{software engineers} shall] \emph{\textbf{assist colleagues} in being fully aware of \textbf{current standard work practices} including \textbf{policies and procedures for protecting passwords, files and other confidential information, and security measures in general}.} \cite{se_code}
\end{quote}

On October 2010, Eric Butler presented as one of these experts. His talk summarized the weaknesses of HTTP, specifically as it relates to web session traffic. It also detailed how HTTPS can be used to effectively eliminate this vulnerability. His stance on educating his peers is aligned perfectly with the position taken by the SE code. As a software engineer, Butler has answered the call to "assist colleagues" by doing this. His colleagues are the men and women who attended \emph{Toorcon}. This presentation was his method of helping his colleagues realize the state of web security. Specifically, he targeted the "policies and procedures for protecting confidential information and security measure in general". The "current standard work practices" for these "policies and procedures" in this situation are listed out in the talk as being the use of site-wide HTTPS/SSL encryption for web sessions.

Here is the SE code transformed with the context specific definitions:

\begin{quote}
  \textbf{7.06} [\textbf{Eric Butler} shall] \emph{\textbf{talk to the Toorcon attendees} [about] being fully aware of \textbf{web security methods} including \textbf{using site-wide HTTPS/SSL encryption for web sessions}.}
\end{quote}

The talk, titled \emph{Hey Web 2.0: Start protecting user privacy instead of pretending to} (and co-authored by Ian Gallagher), started off by describing the problems with HTTP \cite{toorcon_slides}. To put it simply, HTTP relies on cookies for remembering users' passwords that are entered in web forms. These cookies are sent through the air in plain-text, meaning that they can be easily intercepted and read. This gives anyone in a public wifi network the ability to steal the login credentials for someone who is relying on this unencrypted HTTP protocol.

The talk goes on to describe the available solutions to this problem. He pointed out the problems with some of the solutions (avoiding insecure WiFi, using VPNs) and concluded that "end-to-end encryption" is the only viable one. He then introduced \emph{Firesheep}, showing just how serious the problem is in reality. Butler ends the talk with the idea that users should demand this solution to be picked up by the websites that they use, urging his colleagues to join him in adopting practices to help create a more secure web. In his own words, \emph{companies have a responsibility to protect you...DEMAND SSL everywhere!} \cite{toorcon_slides}. The SE code clearly states that software engineers should be encouraging each other to be aware of best practices regarding security measures, and Eric Butler is doing exactly that with the release of \emph{Firesheep}.

\section{Conclusion}
\emph{Firesheep} has created a storm of different opinions regarding the state of web security and the issues of privacy in general when dealing with computers. Due to the highly complex world of web security, there is definitely no purely objective answer to the question on the ethical nature of the tool's release. The point can be made, however, that the software has had an overall positive impact on the current practices of popular websites \cite{github_reaction} \cite{facebook_reaction} \cite{twitter_reaction}. Also, by applying the ACM and IEEE-CS approved \emph{Software Engineering Code of Ethics and Professional Practice} \cite{se_code}, it can be shown that Eric Butler acted ethically as a professional in the field of computing by doing so.

The need for \emph{Firesheep} exists today, but that may not always be the case. As Butler puts it,

\begin{quote}
  \emph{Going forward the metric of Firesheep’s success will quickly change from amount of attention it gains, to the number of sites that adopt proper security. \textbf{True success will be when Firesheep no longer works at all}} \cite{codebutler_blog_2}.
\end{quote}

Hopefully, this is something that can be achieved in the near future, so that users can truly "win".

\section{Personal Notes}
The case of \emph{Firesheep} has been a real interest to me. I first learned about it a few weeks after its release last year, and I have been amazed at how it has led to a gradual shift in the attitudes about web security since then. I work as a web developer at a local startup in San Luis Obispo and have been leading the work on the creation of a new web application with the team there. We consciously made the choice to support full HTTPS encryption for our users, even though it may not have been the top priority for other companies that are similarly small. The process of getting the HTTPS protocol supported was not as difficult as some people have made it out to be. I'm sure that \emph{Firesheep} has no doubt had an indirect impact on the decision to do this, and I think that's pretty awesome.

I attempted to reach Eric Butler at the email address listed on his blog for any further comments during the course of writing this paper, but I have yet to receive a response.

% Bibliography
\end{multicols}
\newpage
\nocite{*}
\bibliographystyle{IEEEannot}
\bibliography{termpaper}
\end{document}
